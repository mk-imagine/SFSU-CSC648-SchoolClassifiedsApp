\section{List of Non-Functional Requirements}
\begin{enumerate}
	\item Application shall be developed, tested and deployed using tools and servers approved by
Class CTO and as agreed in M0
	\item Application shall be optimized for standard desktop/laptop browsers e.g. must render
correctly on the two latest versions of two major browsers
	\item All or selected application functions must render well on mobile devices
	\item Data shall be stored in the database on the team’s deployment server.
	\item No more than 50 concurrent users shall be accessing the application at any time
	\item Privacy of users shall be protected
	\item The language used shall be English (no localization needed)
	\item Application shall be very easy to use and intuitive
	\item Application should follow established architecture patterns
	\item Application code and its repository shall be easy to inspect and maintain
	\item Google analytics shall be used
	\item \underline{No e-mail clients shall be allowed}. Interested users can only message to sellers via in-site
messaging. One round of messaging (from user to seller) is enough for this application
	\item Pay functionality, if any (e.g. paying for goods and services) shall \underline{not be implemented nor
simulated in UI}.
	\item Site security: basic best practices shall be applied (as covered in the class) for main data
items
	\item Media formats shall be standard as used in the market today
	\item Modern SE processes and practices shall be used as specified in the class, including
collaborative and continuous SW development
	\item The application UI (WWW and mobile) shall \underline{prominently} display the following \underline{exact} text on
all pages \textit{"SFSU Software Engineering Project CSC 648-848, Spring 2022. For
Demonstration Only”} \underline{at the top of the WWW page nav bar}. (Important so as to not confuse
this with a real application).
\end{enumerate}